\section{Introduction}

\Kotlin is a modern multiplatform programming language developed by \href{https://www.jetbrains.com/}{Jetbrains} that works on JVM such as \href{www.scala-lang.org}{\texttt{Scala}}.
\Kotlin is completely interoperable with \texttt{Java}\footnote{All classes written in \Kotlin are callable from \texttt{Java} code and vice-versa.} and it is \textit{Object-oriented} with strong elements of \href{https://en.wikipedia.org/wiki/Functional_programming}{functional programming} that make it more powerful than his father \texttt{Java}.
As specified in the \href{https://kotlinlang.org/#why-kotlin}{main page} of the official website, \Kotlin has also the advantages to be \textit{concise}, \textit{safe in nullability}, \textit{expressive} \textit{interoperable} and \textit{multiplatform}.

In addition to this, \Kotlin is now the official \texttt{Android} language, and now supports also \href{https://kotlinlang.org/docs/multiplatform.html}{multiplatform} allowing the developer to write \Kotlin code that can be compiled for \href{https://kotlinlang.org/docs/native-overview.html}{native} platform (including \texttt{Android} and \texttt{iOS}), \texttt{JVM} and \texttt{JavaScript}.

\Go is an open source programming language developed and supported by \texttt{Google}.
It's an \textit{imperative} and \textit{object-oriented} language strongly designed for concurrency thanks to its very easy way to launch process.
The idea of this language is to maintain the run-time efficiency of \texttt{C} but with more readability and usability. Differently from \texttt{C}, \Go has \textit{memory safety}, \textit{garbage collection} and \textit{structural typing} as said by \href{https://en.wikipedia.org/wiki/Go_(programming_language)}{Wikipedia}.

\textbf{Both of this language supports \href{https://en.wikipedia.org/wiki/Coroutine}{coroutines}} as concurrent units of execution. \textit{Coroutines} are lightweight processes that can run over multiple OS threads allowing to save on thread management costs.

Coroutines and threads are very similar but the main difference is that the firsts are \textit{non-preemptive} (or \textit{cooperatives}) differently from the seconds that are typically \textit{preemptive} and scheduled by the OS. Indeed, the execution of a coroutine can be suspended and resumed by the developer, calling some operations, and not by the OS.