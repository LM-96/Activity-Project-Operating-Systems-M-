\section{Introduction}

\subsection{\Kotlin general overview}

\Kotlin is a modern, multi-platform and blended programming language developed by \href{https://www.jetbrains.com/}{Jetbrains} that works on JVM such as \href{www.scala-lang.org}{\texttt{Scala}}.
\Kotlin is completely interoperable with \texttt{Java}\footnote{All classes written in \Kotlin can be callable from \texttt{Java} code and vice-versa.} and it is \textit{object-oriented} with strong elements of \href{https://en.wikipedia.org/wiki/Functional_programming}{functional programming} that make it more powerful than his father \texttt{Java}.
As specified in the \href{https://kotlinlang.org/#why-kotlin}{main page} of the official website, \Kotlin has also the advantages to be \textit{concise}, \textit{safe in nullability}, \textit{expressive}, \textit{asynchronous}, \textit{interoperable} and \textit{multiplatform}.

Furthermore, since 2019, Google has declared \Kotlin as the preferred language for developing \texttt{Android} applications, establishing it as the \textit{de facto} official language. As anticipated, \Kotlin supports also \href{https://kotlinlang.org/docs/multiplatform.html}{multiplatform} allowing the developer to write \Kotlin code that can be compiled for \href{https://kotlinlang.org/docs/native-overview.html}{native} platforms (including \texttt{Android} and \texttt{iOS}), \texttt{JVM} and \texttt{JavaScript}.\\

\begin{lstlisting}[language=kotlin,caption={Exemples of  \Kotlin 's features}]
	// Concise *************************************************************
	val object = Object()
	
	// Safe in nullability *************************************************
	var name: String = "myName"
	var nullableName: String? = null
	// NOT POSSIBLE TO ASSIGN `nullableName = null`
	// The compiler will say "Null can not be a value of a non-null type String"
	
	// Expressive **********************************************************
	fun List<Int>.evenSum() = filter { it % 2 == 0 }.sum()
	
	fun user(block: UserBuilder.() -> Unit): User {
		val user = User();
		user.block()
		return user
	}
	
	fun main() {
		val numbers = listOf(1, 2, 3, 4, 5, 6)
		println("Sum of even numbers: ${numbers.evenSum()}")
		
		val user = user {
			name = "luca"
			surname = "marchegiani"
		}
		println("User: $user")
	}
	
	// Asynchronous ********************************************************
	suspend fun fetchData(): String {
		delay(2000) // Simulating network delay
		return "Data received"
	}
	
	fun main() = runBlocking {
		val result = async { fetchData() }
		println(result.await()) 
	}
	
	// Interoperable *******************************************************
	public class JavaClass {
		public static String greet(String name) {
			return "Hello, " + name + "!";
		}
	}
	
	fun main() {
		val message = JavaClass.greet("Kotlin")
		println(message) // Calls the Java method from Kotlin
	}
\end{lstlisting}

\subsection{\Go general overview}

\Go is an open source programming language developed and supported by \texttt{Google}.
It's an \textit{imperative} and \textit{typed} language that is strongly designed for concurrency thanks to its very easy way to launch process and its efficiency.
The idea of this language is to maintain the run-time efficiency of \texttt{C} but with more readability and usability. Differently from \texttt{C}, \Go has \textit{memory safety}, \textit{garbage collection} and \textit{structural typing} as said by \href{https://en.wikipedia.org/wiki/Go_(programming_language)}{Wikipedia}.

In the last years, \Go also supports mobile platform (\texttt{Android} and \texttt{iOS}) as described in the official \href{https://go.dev/wiki/Mobile}{wiki}, by writing \textit{all-Go native mobile applications} or \textit{SDK applications} with bindings for \texttt{Java} or \texttt{Objective-C}.
There is also a toolkit called \href{https://fyne.io/}{Fyne} that is free and open source that makes easy to build graphical application also for mobile using \Go.
Anyway,  \Go is not widely used for mobile applications since it is primarily designed for backend and system programming. Additionally, \Kotlin is the official language for Android and is also multi-platform.\\

\begin{lstlisting}[language=go,caption={Exemples of  \Go 's features}]
	// Designed for concurrency *******************************************
	func asyncTask() {
		fmt.Println("Running asynchronously!")
	}
	
	func main() {
		go asyncTask()
	}
	
	// Readability and usability ******************************************
	func add(a, b int) int {
		return a + b
	}
	
	func main() {
		nums := []int{1, 2, 3, 4, 5}
		
		for _, num := range nums {
			fmt.Println("Number:", num)
		}
		
		result := add(3, 7)
		fmt.Println("Sum:", result)
	}
	
\end{lstlisting}

\subsection{Fast comparison}

\Go feels like a \emph{modern reinterpretation of C}, emphasizing simplicity and efficiency, while \Kotlin embodies a \emph{contemporary approach to programming} with expressive syntax and power features, with full support to functional programming.

From the concurrency point of view, \textbf{both of this language supports \href{https://en.wikipedia.org/wiki/Coroutine}{coroutines}} as concurrent units of execution. \textit{Coroutines} are lightweight processes that can run over multiple OS threads, allowing to save on thread management costs.

Coroutines and threads are very similar, but the main difference is that the firsts are \textit{non-preemptive} (or \textit{cooperatives}) differently from the seconds that are typically \textit{preemptive} and scheduled by the OS. Indeed, the execution of a coroutine can be suspended and resumed by the developer, calling some operations, and not by the OS. We will go in the details for both of this language.

To conclude this fast overview, you can find the table below that summarizes the main differences between the two languages:

\begin{center}
	\begin{tabular}{|>{\raggedright\arraybackslash}p{3cm}|>{\raggedright\arraybackslash}p{5cm}|>{\raggedright\arraybackslash}p{5cm}|}
		\hline
		\textbf{Feature} & \textbf{Go (Golang)} & \textbf{Kotlin} \\
		\hline
		\textbf{Typing} & Statically typed & Statically typed \\
		\hline
		\textbf{Null Safety} & No built-in null safety & Built-in null safety with nullable and non-nullable types \\
		\hline
		\textbf{Concurrency} & Goroutines for lightweight concurrency & Coroutines for asynchronous programming \\
		\hline
		\textbf{Interoperability} & Limited interoperability with C/C++ & Full interoperability with Java, multiplatform \\
		\hline
		\textbf{Syntax} & Simple and minimalistic & Concise with modern features (lambda, receivers, infix functions) \\
		\hline
		\textbf{Use Cases} & System programming, cloud services & Android development, server-side applications \\
		\hline
		\textbf{Standard Library} & Rich standard library with built-in concurrency support & Rich standard library with extensive collection utilities, all the Java libraries on \Kotlin JVM \\
		\hline
		\textbf{Tooling} & Supported by editors like \textit{VS Code} or \textit{Goland} & Supported by \textit{IntelliJ IDEA} and Android Studio \\
		\hline
		\textbf{Community} & Strong community with a focus on simplicity and performance & Growing community with a focus on modern development practices \\
		\hline
		\textbf{Functional Programming} & Limited support for functional programming & Strong support with higher-order functions, lambdas, and more \\
		\hline
	\end{tabular}
\end{center}